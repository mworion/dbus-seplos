%% Generated by Sphinx.
\def\sphinxdocclass{report}
\documentclass[a4paper,10pt,english]{sphinxmanual}
\ifdefined\pdfpxdimen
   \let\sphinxpxdimen\pdfpxdimen\else\newdimen\sphinxpxdimen
\fi \sphinxpxdimen=.75bp\relax
\ifdefined\pdfimageresolution
    \pdfimageresolution= \numexpr \dimexpr1in\relax/\sphinxpxdimen\relax
\fi
%% let collapsible pdf bookmarks panel have high depth per default
\PassOptionsToPackage{bookmarksdepth=5}{hyperref}

\PassOptionsToPackage{booktabs}{sphinx}
\PassOptionsToPackage{colorrows}{sphinx}

\PassOptionsToPackage{warn}{textcomp}
\usepackage[utf8]{inputenc}
\ifdefined\DeclareUnicodeCharacter
% support both utf8 and utf8x syntaxes
  \ifdefined\DeclareUnicodeCharacterAsOptional
    \def\sphinxDUC#1{\DeclareUnicodeCharacter{"#1}}
  \else
    \let\sphinxDUC\DeclareUnicodeCharacter
  \fi
  \sphinxDUC{00A0}{\nobreakspace}
  \sphinxDUC{2500}{\sphinxunichar{2500}}
  \sphinxDUC{2502}{\sphinxunichar{2502}}
  \sphinxDUC{2514}{\sphinxunichar{2514}}
  \sphinxDUC{251C}{\sphinxunichar{251C}}
  \sphinxDUC{2572}{\textbackslash}
\fi
\usepackage{cmap}
\usepackage[T1]{fontenc}
\usepackage{amsmath,amssymb,amstext}
\usepackage{babel}



\usepackage{tgtermes}
\usepackage{tgheros}
\renewcommand{\ttdefault}{txtt}



\usepackage[Bjarne]{fncychap}
\usepackage{sphinx}

\fvset{fontsize=auto}
\usepackage{geometry}


% Include hyperref last.
\usepackage{hyperref}
% Fix anchor placement for figures with captions.
\usepackage{hypcap}% it must be loaded after hyperref.
% Set up styles of URL: it should be placed after hyperref.
\urlstyle{same}


\usepackage{sphinxmessages}
\setcounter{tocdepth}{1}



\title{MountWizzard4}
\date{Feb 20, 2024}
\release{3.2.6b18}
\author{Michael Würtenberger}
\newcommand{\sphinxlogo}{\sphinxincludegraphics{mw4.png}\par}
\renewcommand{\releasename}{Release}
\makeindex
\begin{document}

\ifdefined\shorthandoff
  \ifnum\catcode`\=\string=\active\shorthandoff{=}\fi
  \ifnum\catcode`\"=\active\shorthandoff{"}\fi
\fi

\pagestyle{empty}
\sphinxmaketitle
\pagestyle{plain}
\sphinxtableofcontents
\pagestyle{normal}
\phantomsection\label{\detokenize{index::doc}}


\sphinxAtStartPar
MW4 is a general utility for 10micron users for improving the workflow for astronomy
work. It runs on Windows11, Windows10 (Win7 should be fine, but it will be not
tested), Mac OSX (beginning from 10.12 to 11.x) including M1 variants if Rosetta is
used and Linux (Ubuntu from 16.04 to 20.04). If you have some knowledge around
Raspberry Pi’s and other SOC, you might be able to install MW4 on a RPi3+, RPi4.


\chapter{Before starting}
\label{\detokenize{index:before-starting}}
\sphinxAtStartPar
First let us have a look to the basic architecture: MW4 is an application
installed on your external computer which is connected to the mount computer via
an IP connection. The best choice is to use a wired connection. As the 10micron
mounts also support a serial line, please be reminded MW4 does not! Many of the
features are handled on the mount computer itself and MW4 does the GUI frontend
for the user by using the command protocol provided by 10micron.

\noindent{\hspace*{\fill}\sphinxincludegraphics[scale=0.71]{{overview}.png}\hspace*{\fill}}

\sphinxAtStartPar
The basic idea is that MW4 will try to generate “digital twin” for the mount. All
parameter changes for the mount will be sent to it and changes of it’s state are
polled to make status visible in MW4. Therefore regular polling of data is needed.


\chapter{Overview}
\label{\detokenize{index:overview}}
\sphinxAtStartPar
Beside this documentation there is a youtube channel available with descriptions,
previews, explanations:

\sphinxAtStartPar
\sphinxurl{https://www.youtube.com/channel/UCJD-5qdLEcBTCugltqw1hXA}

\sphinxAtStartPar
For full operation MW4 supports several frameworks: INDI / INDIGO, ASCOM, Alpaca
and in addition Sequence Generator Pro and N.I.N.A. as camera device.


\chapter{Known limitations}
\label{\detokenize{index:known-limitations}}
\sphinxAtStartPar
MW4 does support python 3.8 \sphinxhyphen{} 3.10 right now. The reason for that is the
lack of precompiled packages. Some features are limited to windows as they need
the original 10micron updater program for execution.

\sphinxAtStartPar
On windows please check if you are working in a 32bit or 64bit environment. You
need to choose the ASCOM setup (drivers etc.) and the python install accordingly.

\sphinxAtStartPar
If you are using the 10micron updater features on windows, MW4 remote controls the
updater application. Please do not interrupt this automation.


\chapter{Reporting issues}
\label{\detokenize{index:reporting-issues}}
\sphinxAtStartPar
To have an eye on your setup here are some topics which you could check:
\begin{itemize}
\item {} 
\sphinxAtStartPar
Mount connection available and stable. Wifi might have performance problems.
Look for right network settings in mount and local setup.

\item {} 
\sphinxAtStartPar
Good counter check is review settings, status bars, message window if something
is going wrong.

\end{itemize}

\sphinxAtStartPar
To improve quality and usability any feedback is highly welcome! To maintain a good
transparency and professional work for my, please respect the following
recommendations how to feed back.

\begin{sphinxadmonition}{note}{Note:}
\sphinxAtStartPar
Please report issues / bugs here:

\sphinxAtStartPar
\sphinxurl{https://github.com/mworion/MountWizzard4/issues}.

\sphinxAtStartPar
And if you have feature requests discussions or for all other topics of
interest there is a good place to start here:

\sphinxAtStartPar
\sphinxurl{https://github.com/mworion/MountWizzard4/discussions}
\end{sphinxadmonition}

\sphinxAtStartPar
In case of a bug report please have a good description (maybe a screenshot if it‘s
related to GUI) and add the log file(s). Normally you just could drop the log file
(or PNG in case of a screen shot) directly to the webpage issues on GitHub. In
some cases GitHub does not accept the file format (unfortunately for example FITs
files). I this case, please zip them and drop the zipped file. This will work. If
you have multiple files, please don‘t zip them to one file! I need them separated
and zipped causes more work.

\sphinxAtStartPar
If changes are made due to a feedback, new releases will have a link to the closed
issues on GitHub.

\sphinxstepscope


\section{Feature Overview}
\label{\detokenize{features:feature-overview}}\label{\detokenize{features::doc}}
\sphinxAtStartPar
For being fully operational, MW4 needs either:
\begin{itemize}
\item {} 
\sphinxAtStartPar
INDI server(s) (see: \sphinxurl{https://indilib.org}) where your devices are connected to.

\item {} 
\sphinxAtStartPar
INDIGO server(s) (see: \sphinxurl{http://www.indigo-astronomy.org}) where your devices
are connected to.

\item {} 
\sphinxAtStartPar
ASCOM Alpaca remote server (see: \sphinxurl{https://ascom-standards.org/FAQs/Index.htm})
abstracting your ASCOM devices or devices which speak native ASCOM Alpaca if
you want to connect over IP with your environment.

\item {} 
\sphinxAtStartPar
Running versions of Sequence Generator Pro or N.I.N.A. as frontend for camera
device.

\item {} 
\sphinxAtStartPar
For the core devices there is native ASCOM support (Windows platform only).
Please be reminded, that ASCOM has 32bit and 64bit driver implementations
and MW4 could also be installed in 32bit or 64 bit python environment. They
could be not be mixed! 32bit python supports only 32bit drivers and vice versa
. Normally this should not be an issue…

\item {} 
\sphinxAtStartPar
In addition an internet connection is used for some services which might be
very helpful.

\end{itemize}

\sphinxAtStartPar
It is recommended to use mount firmware 3.x or later as some of the functions
don’t work with older firmware versions. It should not be a problem using older
versions. A HW pre2012 might also have some issues. MW4 supports also older
firmwares from 2.x onwards, but with limited features and untested.

\noindent{\hspace*{\fill}\sphinxincludegraphics{{versions}.png}\hspace*{\fill}}

\sphinxAtStartPar
It is recommended to use mount firmware 2.16 or later as some of the functions
don’t work with older firmware versions.

\sphinxAtStartPar
Here is an overview of the functionality available in MW4:
\begin{itemize}
\item {} 
\sphinxAtStartPar
Many settings and features of the mount can be shown and changed.

\item {} 
\sphinxAtStartPar
Control movement of the mount as well as tracking speeds.

\item {} 
\sphinxAtStartPar
Coordinates in J2000 as well as in JNow.

\item {} 
\sphinxAtStartPar
Virtual keypad

\item {} 
\sphinxAtStartPar
Model building with different model setups and model generating capabilities.
Sorting points for effective slew paths or dome situations.

\item {} 
\sphinxAtStartPar
Model building is done in parallel threads (imaging, plate solving, slewing)
to reduce time.

\item {} 
\sphinxAtStartPar
Show the actual model and alignment error. Give hints on how to improve the
raw polar alignment.

\item {} 
\sphinxAtStartPar
Model optimisation: deleting points, automatic removing point for target RMS etc.

\item {} 
\sphinxAtStartPar
Manage models stored in the mount (save, load, delete).

\item {} 
\sphinxAtStartPar
Dome geometry integration (MW4 knows about target flip side and slews dome
correctly as well as any geometrical constraints).

\item {} 
\sphinxAtStartPar
Environment data: MW4 shows data from OpenWeatherMap, ClearOutside, External
Sensors like MBox, Stickstation, UniHedronSQR as well as direct linked sensors
like MGBox.

\item {} 
\sphinxAtStartPar
Refraction handling external / internal from the above sources.

\item {} 
\sphinxAtStartPar
Satellite: searching, displaying, programming, updating tracking.

\item {} 
\sphinxAtStartPar
Tools: FITS Header renaming, Park positions, etc.

\item {} 
\sphinxAtStartPar
Remote shutdown of MW4 and Mount via IP commands.

\item {} 
\sphinxAtStartPar
Measurements and CSV saving for most environment and mount data

\item {} 
\sphinxAtStartPar
Imaging: control of connected camera / cooler / filter.

\item {} 
\sphinxAtStartPar
WOL (wake on LAN) boot for mount. MW4 catches MAC address automatically on
first manual start.

\item {} 
\sphinxAtStartPar
Audio signals for different events (end slew, finished modeling, alert, etc.)

\item {} 
\sphinxAtStartPar
Updater for all MW4 functions.

\item {} 
\sphinxAtStartPar
Generate / load / save as many profiles as you would like.

\item {} 
\sphinxAtStartPar
Show alignment stars. Choose and automatically center for polar or orthogonal
adjustments.

\item {} 
\sphinxAtStartPar
Imaging: expose one or N images, auto solve or auto stack these images.

\item {} 
\sphinxAtStartPar
Imaging: show distortion grid, astrometric calculations (flux, roundness,
sharpness)

\end{itemize}

\sphinxstepscope


\section{Installing}
\label{\detokenize{install/index:installing}}\label{\detokenize{install/index::doc}}
\sphinxAtStartPar
If you on the way of installing MW4 to your windows system, please be aware of the
32bit / 64bit limitations of ASCOM / drivers and python. If you are using 64bit
drivers (most likely with the new large scale CMOS cameras), you need to install
64bit python as well as windows does not mix both variants flawless.

\begin{sphinxadmonition}{warning}{Warning:}
\sphinxAtStartPar
I strongly recommend not using whitespace in filenames or directory
paths. Especially in windows handling them is not straight forward
and I hardly could do all the tests needed to ensure it’s
functionality.
\end{sphinxadmonition}

\sphinxstepscope


\subsection{Install Python}
\label{\detokenize{install/python:install-python}}\label{\detokenize{install/python::doc}}
\sphinxAtStartPar
MW4 is a python3 application based on some python libraries and uses Qt as
framework for GUI. Different to past versions of MW there will be no one box
solution (MAC bundle, EXE File, etc.) available. As MW4 is python3 and comes with
internal update functionality, it uses a standard python3 environment. Ideally it
is recommended in a virtualenv.

\sphinxAtStartPar
MW4 is tested on python 3.8 \sphinxhyphen{} 3.10. The first step is to install the python3
.8.x package if not already installed. For all platforms there is an installer
available. Please follow the descriptions that comes with the installers. To give
a short overview here are some quick installation hints for all platforms. The
installers for Windows and OSx can be downloaded from python.org.

\begin{sphinxadmonition}{warning}{Warning:}
\sphinxAtStartPar
Please do not use a newer version of python than 3.9 if you would use MW4 on
other platforms than Windows, Mac or x86 Linux. Some libraries bring
precompiled binaries with them and they might not be available for a newer
python version.
\end{sphinxadmonition}

\sphinxAtStartPar
If you already have python 3.8 \sphinxhyphen{} 3.10 installed, you can skip this section
and go directly to the MW4 installation process. If you have to install python3.8
this has to be done only once for as many MW4 installations you might want.

\sphinxAtStartPar
There is a video on youtube with the install process python: \sphinxurl{https://youtu.be/xJxpx\_SmrVc}.

\begin{sphinxadmonition}{note}{Note:}
\sphinxAtStartPar
On windows there are some new features which supports comet,
earth rotation and asteroids update for the mount. These functions are
available from python 3.8.2 on. Earlier python versions have issues. If you
would like to upgrade an older python installation, please see the comments
below for windows. On other OS there is no need for doing that.
\end{sphinxadmonition}


\subsubsection{Windows}
\label{\detokenize{install/python:windows}}
\begin{sphinxadmonition}{warning}{Warning:}
\sphinxAtStartPar
Windows makes a hard split between 32bit and 64bit versions. If your
drivers and setup uses 64bit solutions, please install 64bit python!
\end{sphinxadmonition}

\sphinxAtStartPar
Depending on your Windows version please download or directly run the web
installer from:

\sphinxAtStartPar
\sphinxurl{https://www.python.org/downloads/windows/}

\sphinxAtStartPar
and follow the installation procedure.

\begin{sphinxadmonition}{warning}{Warning:}
\sphinxAtStartPar
Please take care that during the installation the checkbox “Add Python Path”
is selected and to install for a single user if you want to use the scripts.
\end{sphinxadmonition}

\noindent{\hspace*{\fill}\sphinxincludegraphics[scale=0.71]{{python_win_path}.png}\hspace*{\fill}}

\sphinxAtStartPar
Depending on your preference you could install python 3.10 for a single user or for
all users. MW4 does not need admin rights to run, so please choose the variant for
a single users if you want wo use the installation scripts. They depend on access
rights as a normal user and you might run into troubles using different modes!


\subsubsection{Mac OSx}
\label{\detokenize{install/python:mac-osx}}
\sphinxAtStartPar
Depending on your OSx version please download the installer for 3.10 from:

\sphinxAtStartPar
\sphinxurl{https://www.python.org/downloads/mac-osx/}

\sphinxAtStartPar
and follow the installation procedure. Depending on your preference you could install
python3 for a single user or for all users. MW4 does not need admin rights to run,
so please feel free to choose the variant you would like to use.

\begin{sphinxadmonition}{warning}{Warning:}
\sphinxAtStartPar
Using a Mac with Apple silicon need special treatment. There is rather any
experience with these setups. Actually MW4 only support Intel architecture so
you need to use the Rosetta emulator.
\end{sphinxadmonition}


\subsubsection{Ubuntu}
\label{\detokenize{install/python:ubuntu}}
\sphinxAtStartPar
Referring to Ubuntu 18.04 LTS as it comes with python3.6. This should work, but
you could upgrade to python 3.10. This could be done by adding an appropriate
repo, which enables this version.

\begin{sphinxadmonition}{hint}{Hint:}
\sphinxAtStartPar
If you update to a higher python version, please update to python 3.10 if you
want in a way, which fits best to your environment. There are many
descriptions out, so please search for it in case you don’t know exactly.
\end{sphinxadmonition}

\sphinxAtStartPar
An example is from: \sphinxurl{https://linuxize.com/post/how-to-install-python-3-7-on-ubuntu-18-04/}

\begin{sphinxVerbatim}[commandchars=\\\{\}]
\PYG{n}{sudo} \PYG{n}{add}\PYG{o}{\PYGZhy{}}\PYG{n}{apt}\PYG{o}{\PYGZhy{}}\PYG{n}{repository} \PYG{n}{ppa}\PYG{p}{:}\PYG{n}{deadsnakes}\PYG{o}{/}\PYG{n}{ppa}
\PYG{n}{sudo} \PYG{n}{apt}\PYG{o}{\PYGZhy{}}\PYG{n}{get} \PYG{n}{update}
\PYG{n}{sudo} \PYG{n}{apt}\PYG{o}{\PYGZhy{}}\PYG{n}{get} \PYG{n}{upgrade}
\PYG{n}{sudo} \PYG{n}{apt}\PYG{o}{\PYGZhy{}}\PYG{n}{get} \PYG{n}{install} \PYG{n}{python3}\PYG{l+m+mf}{.10}
\end{sphinxVerbatim}

\sphinxAtStartPar
Please check the right version and the availability of virtualenv in your setup. If
virtualenv is not present in your setup, please install it prior to run the install
scripts with:

\begin{sphinxVerbatim}[commandchars=\\\{\}]
\PYG{n}{sudo} \PYG{n}{apt}\PYG{o}{\PYGZhy{}}\PYG{n}{get} \PYG{n}{install} \PYG{n}{python3}\PYG{o}{\PYGZhy{}}\PYG{n}{virtualenv}
\end{sphinxVerbatim}


\subsubsection{Updating python in your existing environment}
\label{\detokenize{install/python:updating-python-in-your-existing-environment}}
\sphinxAtStartPar
This is a step which should be done if you are familiar with some pc experience.
Hence the steps are not complicated, the setups of you environment might be
somehow special and need a adjusted treatment. The following steps explain a
standard procedure.


\paragraph{Update python version on your windows computer}
\label{\detokenize{install/python:update-python-version-on-your-windows-computer}}
\sphinxAtStartPar
Please go to the python website an download the appropriate python version. On
windows please check the selection of the 32bit or 64bit correctly. It should be
the version you have already chosen.

\sphinxAtStartPar
Start the python installer. If everything went right, it will show an update offer
. If so, please chose that and you get the upgrade. If you would like to switch
from 32bit to 64bit or vice versa, the updater only shows a new install. In this
case please deinstall the old version manually. Than it’s like a new python
installation, please see above.

\sphinxAtStartPar
Having your python version updated on you computer, you have to update the new
version to you work environment(s), too. There are two ways to do that. First you
could use the install script provided and install MW in a new work dir. You could
copy all you settings (except the ‘venv’ folder) to the new workdir. Another way
is to open a command window, change to your work directory and run the command:

\begin{sphinxVerbatim}[commandchars=\\\{\}]
\PYG{n}{python} \PYG{o}{\PYGZhy{}}\PYG{n}{m} \PYG{n}{venv} \PYG{o}{\PYGZhy{}}\PYG{o}{\PYGZhy{}}\PYG{n}{upgrade} \PYG{n}{venv}
\end{sphinxVerbatim}

\sphinxAtStartPar
This will upgrade your work environment to the python version of your computer (so
the updated one)

\begin{sphinxadmonition}{note}{Note:}
\sphinxAtStartPar
Before doing any changes or updates, please do a backup of your environment to
be safe in case of errors in the update process. This could simply be done by
making a copy of your work folder.
\end{sphinxadmonition}

\sphinxstepscope


\subsection{Install MW4}
\label{\detokenize{install/mw4:install-mw4}}\label{\detokenize{install/mw4::doc}}
\sphinxAtStartPar
When starting with the installation of MW4, python 3.8 \sphinxhyphen{} 3.10 should be successful
installed. To check, open a terminal (available on all platforms) and run the
command

\begin{sphinxVerbatim}[commandchars=\\\{\}]
\PYG{n}{python3} \PYG{o}{\PYGZhy{}}\PYG{o}{\PYGZhy{}}\PYG{n}{version}
\PYG{n}{virtualenv} \PYG{o}{\PYGZhy{}}\PYG{o}{\PYGZhy{}}\PYG{n}{version}
\end{sphinxVerbatim}

\sphinxAtStartPar
On windows you can’t call python3, but you have to run the command

\begin{sphinxVerbatim}[commandchars=\\\{\}]
\PYG{n}{python} \PYG{o}{\PYGZhy{}}\PYG{o}{\PYGZhy{}}\PYG{n}{version}
\end{sphinxVerbatim}

\sphinxAtStartPar
In one of the choices you should see the version number of the installed and
available packages. For python it should say 3.8.x … 3.10.x.

\begin{sphinxadmonition}{hint}{Hint:}
\sphinxAtStartPar
MW4 does not need admin rights to install or run. To avoid problems with
accessing directories or file please ensure, that you run install and MW4
itself as normal user!
\end{sphinxadmonition}

\sphinxAtStartPar
To install MW4 on your computer, there are some support available for Windows, OSx
and Ubuntu to make it a little bit easier to install and run MW4. The scripts are
online, and available from Github.


\subsubsection{Installing with installer version 3.x:}
\label{\detokenize{install/mw4:installing-with-installer-version-3-x}}
\sphinxAtStartPar
The install procedure also got improved: You will have only a single compressed
python script (startup.pyz) which is valid for all platforms and does all things
the different existing scripts stand for. Please download the package and unzip it
to get the content. You will find three files:
\begin{itemize}
\item {} 
\sphinxAtStartPar
startup.pyz \sphinxhyphen{}\textgreater{} the script for doing all the work

\item {} 
\sphinxAtStartPar
mountwizzard4.desktop \sphinxhyphen{}\textgreater{} support for ubuntu / linux running the script

\item {} 
\sphinxAtStartPar
mw4.png \sphinxhyphen{}\textgreater{} icon for mountwizzard4.desktop

\item {} 
\sphinxAtStartPar
mw4.ico \sphinxhyphen{}\textgreater{} icon to customize the link in windows for running the script

\end{itemize}

\sphinxAtStartPar
\sphinxurl{https://github.com/mworion/mountwizzard4/blob/master/support/3.0/startupPackage.zip?raw=true}

\sphinxAtStartPar
On windows you should be able to start the script just be double click on it,
in all other platforms you start it with:

\begin{sphinxVerbatim}[commandchars=\\\{\}]
\PYG{n}{python3} \PYG{n}{startup}\PYG{o}{.}\PYG{n}{pyz}
\end{sphinxVerbatim}

\sphinxAtStartPar
On windows please use the command:

\begin{sphinxVerbatim}[commandchars=\\\{\}]
\PYG{n}{python} \PYG{n}{startup}\PYG{o}{.}\PYG{n}{pyz}
\end{sphinxVerbatim}

\begin{sphinxadmonition}{warning}{Warning:}
\sphinxAtStartPar
The new script 3.x supports only Windows, Mac and x86 Linux distributions!
This is valid also for the support of MW4 v3.x
If you need support other platforms, please use the actual scripts 2.x. and
MW4 v2.x versions.
\end{sphinxadmonition}


\subsubsection{Downloading the zip files}
\label{\detokenize{install/mw4:downloading-the-zip-files}}
\sphinxAtStartPar
Please click the link and press download from the page:

\noindent{\hspace*{\fill}\sphinxincludegraphics[scale=0.71]{{scripts}.png}\hspace*{\fill}}

\sphinxAtStartPar
There is a video on youtube with the install process for Mac:
\sphinxurl{https://youtu.be/xJxpx\_SmrVc}.


\subsubsection{Short videos for installation}
\label{\detokenize{install/mw4:short-videos-for-installation}}
\sphinxAtStartPar
For a better impression of how MW4 could be installed, there are some special
videos showing a installation on different platforms.
\begin{multicols}{1}\raggedright
\begin{itemize}\setlength{\itemsep}{0pt}\setlength{\parskip}{0pt}
\item {} 
\sphinxAtStartPar
Windows10: \sphinxurl{https://youtu.be/q9WbiHhW5NU}

\item {} 
\sphinxAtStartPar
Mac OS Catalina: \sphinxurl{https://youtu.be/bbZ9\_yLm1TU}

\item {} 
\sphinxAtStartPar
Ubuntu 18.04: \sphinxurl{https://youtu.be/kNfLrtJtkq8}

\end{itemize}\raggedcolumns\end{multicols}


\subsubsection{Step 1}
\label{\detokenize{install/mw4:step-1}}
\sphinxAtStartPar
Please create a working directory of your choice and location. For MacOSx I would
recommend not using a location on the desktop as it might cause troubles with
execution right in newer OSx installations. The directory can be renamed later on,
it also can also be moved to any other location. Copy the scripts for your
platform into this directory.

\begin{sphinxadmonition}{hint}{Hint:}
\sphinxAtStartPar
Over time, there might be some improvements also made for these scripts.
So if you had installed MW4 some time ago and will install new setups,
it might be helpful to check if some new scripts are available for better
handling.
\end{sphinxadmonition}

\sphinxAtStartPar
the directory should than for OSx look like:

\noindent{\hspace*{\fill}\sphinxincludegraphics[scale=0.71]{{mac_1}.png}\hspace*{\fill}}

\sphinxAtStartPar
In Windows10 it looks like:

\noindent{\hspace*{\fill}\sphinxincludegraphics[scale=0.71]{{win_1}.png}\hspace*{\fill}}

\begin{sphinxadmonition}{warning}{Warning:}
\sphinxAtStartPar
Please closely check if your working directory is writable. Otherwise MW4 could
not work properly!
\end{sphinxadmonition}

\noindent{\hspace*{\fill}\sphinxincludegraphics[scale=0.71]{{win_1}.png}\hspace*{\fill}}

\sphinxAtStartPar
Windows10 might as you the first time of execution the following question:

\noindent{\hspace*{\fill}\sphinxincludegraphics[scale=0.71]{{win_a}.png}\hspace*{\fill}}

\sphinxAtStartPar
and you could accept that by clicking “addition information” and than execute:

\noindent{\hspace*{\fill}\sphinxincludegraphics[scale=0.71]{{win_b}.png}\hspace*{\fill}}


\subsubsection{Step 2}
\label{\detokenize{install/mw4:step-2}}
\sphinxAtStartPar
Run one of the scripts following script. During installation a terminal window
might and shows the progress of installation.

\begin{sphinxVerbatim}[commandchars=\\\{\}]
\PYG{n}{MW4\PYGZus{}Install}\PYG{o}{.}\PYG{n}{bat}         \PYG{c+c1}{\PYGZsh{} Windows}
\PYG{n}{MW4\PYGZus{}Install}\PYG{o}{.}\PYG{n}{sh}          \PYG{c+c1}{\PYGZsh{} Ubuntu}
\PYG{n}{MW4\PYGZus{}Install}\PYG{o}{.}\PYG{n}{command}     \PYG{c+c1}{\PYGZsh{} OSx}
\end{sphinxVerbatim}

\sphinxAtStartPar
With the script a virtual environment for python is installed in your working dir
under the name “venv”. After that it installs all necessary libraries and MW4
itself into this virtual environment. So any other installation of python
applications is not influenced by MW4 install.

\sphinxAtStartPar
After running the install script the directory should for OSx look like:

\noindent{\hspace*{\fill}\sphinxincludegraphics[scale=0.71]{{mac_2}.png}\hspace*{\fill}}

\sphinxAtStartPar
In Windows10 it looks like:

\noindent{\hspace*{\fill}\sphinxincludegraphics[scale=0.71]{{win_2}.png}\hspace*{\fill}}

\sphinxAtStartPar
In Windows10 for the first time you might be asked again for permission (see above).

\sphinxAtStartPar
Please use for the following step the install marked in red.

\sphinxAtStartPar
MW4 is already installed inside the virtual environment venv in your work dir.

\begin{sphinxadmonition}{warning}{Warning:}
\sphinxAtStartPar
Please check if an online connection is available on your computer during
installation as the libraries and MW4 is installed from online sources.
\end{sphinxadmonition}


\subsubsection{Step 3}
\label{\detokenize{install/mw4:step-3}}
\sphinxAtStartPar
Run one of the scripts

\begin{sphinxVerbatim}[commandchars=\\\{\}]
\PYG{n}{MW4\PYGZus{}Run}\PYG{o}{.}\PYG{n}{bat}         \PYG{c+c1}{\PYGZsh{} Windows}
\PYG{n}{MW4\PYGZus{}Run}\PYG{o}{.}\PYG{n}{sh}          \PYG{c+c1}{\PYGZsh{} Ubuntu}
\PYG{n}{MW4\PYGZus{}Run}\PYG{o}{.}\PYG{n}{command}     \PYG{c+c1}{\PYGZsh{} OSx}
\end{sphinxVerbatim}

\sphinxAtStartPar
This script will start MW4 for the first time and it will create some
subdirectories in your working folder. When starting, a splash screen show the
progress of it’s initialization. After first start the directory should for OSx
look like:

\noindent{\hspace*{\fill}\sphinxincludegraphics[scale=0.71]{{mac_3}.png}\hspace*{\fill}}

\sphinxAtStartPar
In Windows10 it looks like:

\noindent{\hspace*{\fill}\sphinxincludegraphics[scale=0.71]{{win_3}.png}\hspace*{\fill}}

\sphinxAtStartPar
In Windows10 for the first time you might be asked again for permission (see above).

\sphinxAtStartPar
With the first run you will see a log file written and you should have a first
window from MW4 open. Please notice that there will be no visible terminal window,
but a minimized power shell in the menu. This might take some seconds before MW4
comes up with the splash screen:

\noindent{\hspace*{\fill}\sphinxincludegraphics[scale=0.71]{{first_run}.png}\hspace*{\fill}}

\sphinxAtStartPar
If you see the upper window, you succeed and from now on you are able to customize your
setup of MW4 and it’s features.


\subsubsection{Setting up Ubuntu}
\label{\detokenize{install/mw4:setting-up-ubuntu}}
\sphinxAtStartPar
For Ubuntu the scripts also include an icon file (mw4.png) as well as a desktop
description file (MountWizzard4.desktop). In order to use this add\sphinxhyphen{}on, please
adjust the directories used in this file:

\noindent{\hspace*{\fill}\sphinxincludegraphics[scale=0.71]{{ubuntu_setup}.png}\hspace*{\fill}}

\sphinxAtStartPar
Unfortunately this is broken un Ubuntu 20.04LTS, see (including the workaround):

\sphinxAtStartPar
\sphinxurl{https://askubuntu.com/questions/1231413/basic-desktop-actions-are-not-available-on-ubuntu-20-04}

\sphinxAtStartPar
If you install nemo (hint as workaround) as file manager, the desktop icons will work.


\subsubsection{DPI scaling on Windows}
\label{\detokenize{install/mw4:dpi-scaling-on-windows}}
\sphinxAtStartPar
If you are running a windows machine with setting the zoom factor for you display
settings different to 100\%, you might notice inadequate font sizes etc.
Unfortunately this could not be worked around within MW4 itself, but you could
change some environment variables to omit this problem. The actual script already
contain some setting to keep the resolution to 100\% even if you choose to increase
this value for other applications. You want to play with these settings to make
the appearance correct:

\begin{sphinxVerbatim}[commandchars=\\\{\}]
\PYG{n}{SET} \PYG{n}{QT\PYGZus{}SCALE\PYGZus{}FACTOR}\PYG{o}{=}\PYG{l+m+mi}{1}
\PYG{n}{SET} \PYG{n}{QT\PYGZus{}FONT\PYGZus{}DPI}\PYG{o}{=}\PYG{l+m+mi}{96}
\end{sphinxVerbatim}

\sphinxAtStartPar
Here some examples of the settings: Normal scaling (scale = 1, dpi = 96)

\noindent{\hspace*{\fill}\sphinxincludegraphics[scale=0.71]{{scale_normal}.png}\hspace*{\fill}}

\sphinxAtStartPar
Small fonts (scale = 1, dpi = 48)

\noindent{\hspace*{\fill}\sphinxincludegraphics[scale=0.71]{{scale_dpi48}.png}\hspace*{\fill}}

\sphinxAtStartPar
Bigger scale (scale = 1.5, dpi = 96)

\noindent{\hspace*{\fill}\sphinxincludegraphics[scale=0.71]{{scale_1_5}.png}\hspace*{\fill}}

\sphinxAtStartPar
If you would like to have MW4 displayed bigger than 100\%, please increase the
QT\_SCALE\_FACTOR to the value desired. A value of 1 means 100\%, so 2 means 200\%.
You will experience to set the font adequately.


\subsubsection{DPI scaling on Ubuntu}
\label{\detokenize{install/mw4:dpi-scaling-on-ubuntu}}
\sphinxAtStartPar
This is quite similar to windows. You have to set the environment variables
QT\_SCALE\_FACTOR and QT\_FONT\_DPI accordingly. They are already part of the
MW4\_Run.sh scripts.


\subsubsection{Installation on Apple Silicon}
\label{\detokenize{install/mw4:installation-on-apple-silicon}}
\sphinxAtStartPar
For software that is not yet updated, Apple has built in translation software
called Rosetta 2. Rosetta 2 will interpret  traditional Intel\sphinxhyphen{}based code and make
it look like ARM\sphinxhyphen{}based code. And it does this pretty well. Generally speaking as a
user it is very difficult to distinguish between apps that have ‘native M1
support’ to traditional Intel\sphinxhyphen{}based apps.

\sphinxAtStartPar
But for any apps that are run from the command\sphinxhyphen{}line in Terminal, this standard
Rosetta 2 translation does not happen. Within Astrophotography it is not uncommon
to have apps that run from the command\sphinxhyphen{}line. Please hav a look to:
\sphinxurl{https://www.astroworldcreations.com/blog/apple-silicon-and-legacy-command-line-software}


\subsubsection{Update manually}
\label{\detokenize{install/mw4:update-manually}}
\sphinxAtStartPar
If you plan to upgrade MW4 to the newest release, MW4 has it’s own internal
updater and using the script is not necessary. In some circumstances this might
be necessary. In these cases you could use on of the

\begin{sphinxVerbatim}[commandchars=\\\{\}]
\PYG{n}{MW4\PYGZus{}Update}\PYG{o}{.}\PYG{n}{bat}         \PYG{c+c1}{\PYGZsh{} Windows}
\PYG{n}{MW4\PYGZus{}Update}\PYG{o}{.}\PYG{n}{sh}          \PYG{c+c1}{\PYGZsh{} Ubuntu}
\PYG{n}{MW4\PYGZus{}Update}\PYG{o}{.}\PYG{n}{command}     \PYG{c+c1}{\PYGZsh{} OSx}
\end{sphinxVerbatim}

\sphinxAtStartPar
scripts. The command script updates to the latest release.

\begin{sphinxadmonition}{note}{Note:}
\sphinxAtStartPar
You only could update to official releases. Beta’s are not supported.
\end{sphinxadmonition}

\sphinxstepscope


\subsection{Install on Astroberry RaspberryPi}
\label{\detokenize{install/astroberry:install-on-astroberry-raspberrypi}}\label{\detokenize{install/astroberry::doc}}
\sphinxAtStartPar
I strongly recommend using the actual astroberry server \sphinxurl{https://www.astroberry.io}
if you would like to use KSTars/EKOS and MountWizzard4 on a Raspi (3 or 4).

\sphinxAtStartPar
Under \sphinxurl{https://github.com/mworion/MountWizzard4/tree/master/support} you will find
the necessary scripts to install MW4 directly under astroberry. Please download
the corresponding ZIP archive (Astroberry\_Scripts.zip), make a working directory
on astroberry server, extract the archive there and run the install script.

\sphinxAtStartPar
For starting MW4, please start the run script. Basically that’s all to do.

\begin{sphinxadmonition}{hint}{Hint:}
\sphinxAtStartPar
Actually astroberry is supported from MW4 version 2.1.3 on. It has some
limitations in features: no support of 3D mount simulator.
\end{sphinxadmonition}

\begin{sphinxadmonition}{warning}{Warning:}
\sphinxAtStartPar
The installation for is dedicated for a 32bit system on Raspi4. If
you plan to use a 64 bis system, please look to StellarMate64
\end{sphinxadmonition}

\sphinxstepscope


\subsection{Install on StellarMate1.7++ (64bit)}
\label{\detokenize{install/stellarmate64:install-on-stellarmate1-7-64bit}}\label{\detokenize{install/stellarmate64::doc}}
\sphinxAtStartPar
Another way of using MW4 on Raspi4 64 Bit is to download bullseye 64bit image or
an Stellarmate1.7++ image. On the support page you will find the corresponding
install and run script as ZIP archive. Please download them in your work folder.
The scripts will use precompiled wheels for aarch64 as much as possible to improved
the installation speed e.g. on your RPi4.

\begin{sphinxVerbatim}[commandchars=\\\{\}]
\PYG{o}{.}\PYG{o}{/}\PYG{n}{MW4\PYGZus{}Install}\PYG{o}{.}\PYG{n}{sh}
\end{sphinxVerbatim}

\sphinxAtStartPar
After a short while MW4 is installed and should be ready to run like in ubuntu
installation.

\begin{sphinxadmonition}{note}{Note:}
\sphinxAtStartPar
The install scripts only support python 3.8\sphinxhyphen{}3.9 versions”
\end{sphinxadmonition}

\sphinxstepscope


\subsection{Install on RaspberryPi 3}
\label{\detokenize{install/rpi3:install-on-raspberrypi-3}}\label{\detokenize{install/rpi3::doc}}
\begin{sphinxadmonition}{hint}{Hint:}
\sphinxAtStartPar
The simplest raspi installation for rpi3 works with astroberry
\end{sphinxadmonition}


\subsubsection{Installing Python on RPi3}
\label{\detokenize{install/rpi3:installing-python-on-rpi3}}
\sphinxAtStartPar
To get MW4 installed on RPi3 you will follow the instructions of Robert Lancaste
(many thanks to him fore this work!) on \sphinxurl{https://github.com/rlancaste/AstroPi3} with
installing AstroPi3 scripts. The installation procedure I describe is based on
Raspbian Buster with desktop.
should give you the following result:

\noindent{\hspace*{\fill}\sphinxincludegraphics[scale=0.71]{{rpi3_rlancaste}.png}\hspace*{\fill}}

\sphinxAtStartPar
In addition you have to take care, that python3.8 is installed. The
actual Ubuntu mate 18.04.2 distribution comes with python 3.6, so we need to
update this. Please follow the description: {\hyperref[\detokenize{install/python:ubuntu}]{\sphinxcrossref{\DUrole{std,std-ref}{Ubuntu}}}} (\autopageref*{\detokenize{install/python:ubuntu}}). After that you should
get an python3.8 or newer available on your system:

\noindent{\hspace*{\fill}\sphinxincludegraphics[scale=0.71]{{rpi3_python37}.png}\hspace*{\fill}}

\sphinxAtStartPar
If everything went fine, we can proceed to the next step.


\subsubsection{Installing PyQt5 on RPi3}
\label{\detokenize{install/rpi3:installing-pyqt5-on-rpi3}}
\sphinxAtStartPar
As on arm the installation of PyQt5 could not be done through pip, the actual
tested path is to install Qt directly via apt\sphinxhyphen{}get on your RBP3. As result, you
cannot install MW4 easily in a virtual environment as apt\sphinxhyphen{}get will install all
libraries in a system path.

\sphinxAtStartPar
As there were no compiled binaries for actual Qt version available, you have to
compile it yourself.

\begin{sphinxVerbatim}[commandchars=\\\{\}]
\PYG{n}{sudo} \PYG{n}{apt}\PYG{o}{\PYGZhy{}}\PYG{n}{get} \PYG{n}{update}
\PYG{n}{sudo} \PYG{n}{apt}\PYG{o}{\PYGZhy{}}\PYG{n}{get} \PYG{n}{install} \PYG{n}{python3}\PYG{l+m+mf}{.8}\PYG{o}{\PYGZhy{}}\PYG{n}{dev}
\PYG{n}{sudo} \PYG{n}{apt}\PYG{o}{\PYGZhy{}}\PYG{n}{get} \PYG{n}{install} \PYG{n}{qt5}\PYG{o}{\PYGZhy{}}\PYG{n}{default}
\PYG{n}{sudo} \PYG{n}{apt}\PYG{o}{\PYGZhy{}}\PYG{n}{get} \PYG{n}{install} \PYG{n}{sip}\PYG{o}{\PYGZhy{}}\PYG{n}{dev}

\PYG{n}{cd} \PYG{o}{/}\PYG{n}{usr}\PYG{o}{/}\PYG{n}{src}
\PYG{n}{sudo} \PYG{n}{wget} \PYG{n}{https}\PYG{p}{:}\PYG{o}{/}\PYG{o}{/}\PYG{n}{www}\PYG{o}{.}\PYG{n}{riverbankcomputing}\PYG{o}{.}\PYG{n}{com}\PYG{o}{/}\PYG{n}{static}\PYG{o}{/}\PYG{n}{Downloads}\PYG{o}{/}\PYG{n}{sip}\PYG{o}{/}\PYG{l+m+mf}{4.19}\PYG{l+m+mf}{.19}\PYG{o}{/}\PYG{n}{sip}\PYG{o}{\PYGZhy{}}\PYG{l+m+mf}{4.19}\PYG{l+m+mf}{.19}\PYG{o}{.}\PYG{n}{tar}\PYG{o}{.}\PYG{n}{gz}
\PYG{n}{sudo} \PYG{n}{wget} \PYG{n}{https}\PYG{p}{:}\PYG{o}{/}\PYG{o}{/}\PYG{n}{www}\PYG{o}{.}\PYG{n}{riverbankcomputing}\PYG{o}{.}\PYG{n}{com}\PYG{o}{/}\PYG{n}{static}\PYG{o}{/}\PYG{n}{Downloads}\PYG{o}{/}\PYG{n}{PyQt5}\PYG{o}{/}\PYG{l+m+mf}{5.13}\PYG{l+m+mf}{.2}\PYG{o}{/}\PYG{n}{PyQt5}\PYG{o}{\PYGZhy{}}\PYG{l+m+mf}{5.13}\PYG{l+m+mf}{.2}\PYG{o}{.}\PYG{n}{tar}\PYG{o}{.}\PYG{n}{gz}

\PYG{n}{sudo} \PYG{n}{tar} \PYG{n}{xzf} \PYG{n}{sip}\PYG{o}{\PYGZhy{}}\PYG{l+m+mf}{4.19}\PYG{l+m+mf}{.19}\PYG{o}{.}\PYG{n}{tar}\PYG{o}{.}\PYG{n}{gz}
\PYG{n}{sudo} \PYG{n}{tar} \PYG{n}{xzf} \PYG{n}{PyQt5}\PYG{o}{\PYGZhy{}}\PYG{l+m+mf}{5.13}\PYG{l+m+mf}{.2}\PYG{o}{.}\PYG{n}{tar}\PYG{o}{.}\PYG{n}{gz}

\PYG{n}{cd} \PYG{n}{sip}\PYG{o}{\PYGZhy{}}\PYG{l+m+mf}{4.19}\PYG{l+m+mf}{.19}
\PYG{n}{sudo} \PYG{n}{python3}\PYG{l+m+mf}{.7} \PYG{n}{configure}\PYG{o}{.}\PYG{n}{py} \PYG{o}{\PYGZhy{}}\PYG{o}{\PYGZhy{}}\PYG{n}{sip}\PYG{o}{\PYGZhy{}}\PYG{n}{module} \PYG{n}{PyQt5}\PYG{o}{.}\PYG{n}{sip}
\PYG{n}{sudo} \PYG{n}{make} \PYG{o}{\PYGZhy{}}\PYG{n}{j4}
\PYG{n}{sudo} \PYG{n}{make} \PYG{n}{install}

\PYG{n}{cd} \PYG{n}{PyQt5\PYGZus{}gpl}\PYG{o}{\PYGZhy{}}\PYG{l+m+mf}{5.13}\PYG{l+m+mf}{.2}
\PYG{n}{sudo} \PYG{n}{python3}\PYG{l+m+mf}{.7} \PYG{n}{configure}\PYG{o}{.}\PYG{n}{py}
\PYG{n}{sudo} \PYG{n}{make} \PYG{o}{\PYGZhy{}}\PYG{n}{j4}
\PYG{n}{sudo} \PYG{n}{make} \PYG{n}{install}
\end{sphinxVerbatim}

\sphinxAtStartPar
There are in different packages to be downloaded and installed. They build on each
other, so keep the order of compiling and install. This procedure take about 2
hours or more, depending on the system.

\begin{sphinxadmonition}{warning}{Warning:}
\sphinxAtStartPar
So far PyQtWebEngine does not build on RPi3! So I removed for the build from 0
.138 on the capabilities, who need the PyQtWebEngine package. This is
basically the Keypad. So you will have limited features!
\end{sphinxadmonition}

\sphinxAtStartPar
So before you could actually run MW4 you need to install some mor libraries:

\begin{sphinxVerbatim}[commandchars=\\\{\}]
\PYG{n}{sudo} \PYG{n}{apt}\PYG{o}{\PYGZhy{}}\PYG{n}{get} \PYG{n}{install} \PYG{n}{libgfortran5}
\PYG{n}{sudo} \PYG{n}{apt}\PYG{o}{\PYGZhy{}}\PYG{n}{get} \PYG{n}{install} \PYG{n}{libjpeg}\PYG{o}{\PYGZhy{}}\PYG{n}{dev} \PYG{n}{zlib1g}\PYG{o}{\PYGZhy{}}\PYG{n}{dev}
\PYG{n}{python3}\PYG{l+m+mf}{.7} \PYG{o}{\PYGZhy{}}\PYG{n}{m} \PYG{n}{pip} \PYG{n}{install} \PYG{o}{\PYGZhy{}}\PYG{n}{U} \PYG{n}{Pillow}
\end{sphinxVerbatim}

\sphinxAtStartPar
Once you are set, make a work directory, cd to this directory and install MW4 by

\begin{sphinxVerbatim}[commandchars=\\\{\}]
\PYG{n}{python3}\PYG{l+m+mf}{.8} \PYG{o}{\PYGZhy{}}\PYG{n}{m} \PYG{n}{pip} \PYG{n}{install} \PYG{n}{mountwizzard4}
\end{sphinxVerbatim}

\sphinxAtStartPar
and run MW4 with the command

\begin{sphinxVerbatim}[commandchars=\\\{\}]
\PYG{n}{python3}\PYG{l+m+mf}{.8} \PYG{o}{\PYGZti{}}\PYG{o}{/}\PYG{o}{.}\PYG{n}{local}\PYG{o}{/}\PYG{n}{lib}\PYG{o}{/}\PYG{n}{python3}\PYG{l+m+mf}{.8}\PYG{o}{/}\PYG{n}{site}\PYG{o}{\PYGZhy{}}\PYG{n}{packages}\PYG{o}{/}\PYG{n}{mw4}\PYG{o}{/}\PYG{n}{loader}\PYG{o}{.}\PYG{n}{py}
\end{sphinxVerbatim}

\sphinxAtStartPar
If everything went fine, you should see MW4 on RPi3:

\noindent{\hspace*{\fill}\sphinxincludegraphics[scale=0.71]{{rpi3_running}.png}\hspace*{\fill}}

\sphinxstepscope


\subsection{Install on RaspberryPi 4}
\label{\detokenize{install/rpi4:install-on-raspberrypi-4}}\label{\detokenize{install/rpi4::doc}}
\begin{sphinxadmonition}{hint}{Hint:}
\sphinxAtStartPar
The simplest raspi installation for rpi3 works with astroberry
\end{sphinxadmonition}

\sphinxAtStartPar
We are installing MW4 on an ubuntu 20.04.1LTS 64Bit system. In relation to the
RPi3 it seems to be much simpler to do. Nevertheless some of the big packages will
be compiled on your system during installation, which means this will take some
time (hours). There is the opportunity to use precompiled packages out of the
install scripts provided.

\sphinxAtStartPar
Another big step forward is that you could use now a virtual environment for
installing MW4.


\subsubsection{Installing Python on RPi4}
\label{\detokenize{install/rpi4:installing-python-on-rpi4}}
\sphinxAtStartPar
To get MW4 installed on RPi4 you will follow the instructions of Dustin Casto:

\sphinxAtStartPar
\sphinxurl{https://homenetworkguy.com/how-to/install-ubuntu-mate-20-04-lts-on-raspberry-pi-4/}

\sphinxAtStartPar
to get Ubuntu Mate 20.04.1 LTS on your RPi4.

\begin{sphinxadmonition}{hint}{Hint:}
\sphinxAtStartPar
Some users experience problems with KStars/EKOS on original ubuntu\sphinxhyphen{}mate
desktop. So the recommendation is to use a KDE bases desktop like kubuntu. The
easiest way to install a desktop on top of the server installation is using:
\sphinxurl{https://github.com/wimpysworld/desktopify}
\end{sphinxadmonition}

\sphinxAtStartPar
After you have finished the setup and got the desktop up and running, the command

\begin{sphinxVerbatim}[commandchars=\\\{\}]
\PYG{n}{python3} \PYG{o}{\PYGZhy{}}\PYG{o}{\PYGZhy{}}\PYG{n}{version}
\end{sphinxVerbatim}

\sphinxAtStartPar
should give you the following result 3.8.5: Please take care, that a python
version 3.8.5 or later is installed.

\sphinxAtStartPar
The actual Ubuntu mate 20.04.1LTS distribution comes with python 3.8.5, so
everything should be OK. Next we have to do is to install a virtual environment
capability, the packet manager pip and the development headers for python to be
able to compile necessary packages:

\begin{sphinxVerbatim}[commandchars=\\\{\}]
\PYG{n}{sudo} \PYG{n}{apt}\PYG{o}{\PYGZhy{}}\PYG{n}{get} \PYG{n}{install} \PYG{n}{python3}\PYG{l+m+mf}{.8}\PYG{o}{\PYGZhy{}}\PYG{n}{venv}
\PYG{n}{sudo} \PYG{n}{apt}\PYG{o}{\PYGZhy{}}\PYG{n}{get} \PYG{n}{install} \PYG{n}{python3}\PYG{o}{\PYGZhy{}}\PYG{n}{pip}
\PYG{n}{sudo} \PYG{n}{apt}\PYG{o}{\PYGZhy{}}\PYG{n}{get} \PYG{n}{install} \PYG{n}{qt5}\PYG{o}{\PYGZhy{}}\PYG{n}{default}
\end{sphinxVerbatim}

\begin{sphinxadmonition}{note}{Note:}
\sphinxAtStartPar
You need to have both packages installed as otherwise the install script or
later does MW4 not run.
\end{sphinxadmonition}


\subsubsection{Using the precompiled wheels}
\label{\detokenize{install/rpi4:using-the-precompiled-wheels}}
\sphinxAtStartPar
Please choose the script fitting to you ubuntu version (18.04.x or 20.04.x)
The scripts will use precompiled wheels for aarch64 as much as possible to improved
the installation speed e.g. on your RPi4.

\begin{sphinxVerbatim}[commandchars=\\\{\}]
\PYG{o}{.}\PYG{o}{/}\PYG{n}{MW4\PYGZus{}Install\PYGZus{}aarch64\PYGZus{}18\PYGZus{}04}\PYG{o}{.}\PYG{n}{sh}

\PYG{o+ow}{or}

\PYG{o}{.}\PYG{o}{/}\PYG{n}{MW4\PYGZus{}Install\PYGZus{}aarch64\PYGZus{}20\PYGZus{}04}\PYG{o}{.}\PYG{n}{sh}
\end{sphinxVerbatim}

\sphinxAtStartPar
After a short while MW4 is installed and should be ready to run like in ubuntu
installation.

\begin{sphinxadmonition}{note}{Note:}
\sphinxAtStartPar
The install scripts only support python 3.8\sphinxhyphen{}3.9 versions”
\end{sphinxadmonition}

\sphinxstepscope


\subsection{Using the internal updater}
\label{\detokenize{install/update:using-the-internal-updater}}\label{\detokenize{install/update::doc}}
\sphinxAtStartPar
From version 2 on MW4 brings an improved updater concept. Basically you choose
to update within MW4 ans select the version of choice. After that a check if
version exists will be processed and if successful, MW4 will be closed and the
updater program will be started.

\begin{sphinxadmonition}{hint}{Hint:}
\sphinxAtStartPar
See also on youtube: \sphinxurl{https://youtu.be/TxxnMizbU1g}
\end{sphinxadmonition}

\begin{sphinxadmonition}{note}{Note:}
\sphinxAtStartPar
MW4 does not store your actual settings. If you would like to do so,
please \sphinxstylestrong{save} your configuration before updating.
\end{sphinxadmonition}

\noindent{\hspace*{\fill}\sphinxincludegraphics[scale=0.71]{{update1}.png}\hspace*{\fill}}

\sphinxAtStartPar
Once the updater program has started you see which version will be installed. Now
you could start the update or just cancel it.

\noindent{\hspace*{\fill}\sphinxincludegraphics[scale=0.71]{{update2}.png}\hspace*{\fill}}

\sphinxAtStartPar
If started, the updater will download and install all necessary libraries.

\noindent{\hspace*{\fill}\sphinxincludegraphics[scale=0.71]{{update3}.png}\hspace*{\fill}}

\sphinxAtStartPar
The updater finishes automatically and reloads MW4. This will happen in both cases
\sphinxstylestrong{Cancel Update} and \sphinxstylestrong{Start Update}. If you close the updater
window
directly, MW4 will be not reloaded.

\noindent{\hspace*{\fill}\sphinxincludegraphics[scale=0.71]{{update4}.png}\hspace*{\fill}}

\sphinxAtStartPar
MW4 new version starts.

\sphinxstepscope


\subsection{Install Plate Solvers}
\label{\detokenize{install/platesolvers:install-plate-solvers}}\label{\detokenize{install/platesolvers::doc}}
\sphinxAtStartPar
Supported platesolvers are astrometry.net, ASTAP and Watney. All solvers should be
installed locally. There is no support for astrometry.net online. If you install a
plate solver, please be reminded that you have to install their index files as
well. Unfortunately all are using different index files and methods, also
depending on your optical setup. MW4 helps you in finding the necessary index
files for your setup:

\noindent{\hspace*{\fill}\sphinxincludegraphics[scale=0.71]{{information}.png}\hspace*{\fill}}

\sphinxAtStartPar
Based on optical and Sensor data. MW4 will make a prognosis, which index selection
might be good. You also have direct internet links to the sources.


\subsubsection{Astrometry.net}
\label{\detokenize{install/platesolvers:astrometry-net}}
\sphinxAtStartPar
Astrometry.net is useful on Linux and Mac installations. On Windows there is no
good setup possible. There are many solutions available (e.g. ANSRV), but these
could not be used through MW4. You will find astrometry.net here:

\sphinxAtStartPar
\sphinxurl{http://astrometry.net}


\subsubsection{ASTAP}
\label{\detokenize{install/platesolvers:astap}}
\sphinxAtStartPar
ASTAP is an application available for all platforms from Han. Great software! It
is available as application with GUI and as pure command line interface solution
(CLI). If you don’t use ASTAP for other things as well, I would recommend
installing the CLI version locally in a folder of your choice. You find all
information on:

\sphinxAtStartPar
\sphinxurl{https://www.hnsky.org/astap.htm}


\subsubsection{Watney}
\label{\detokenize{install/platesolvers:watney}}
\sphinxAtStartPar
Watney is a new solver from Jusas, based in the algorithm Han published for ASTAP.
It is available as API and command line interface (CLI) version for all platforms.
For the use with MW4, please install the CLI version in a folder of your choice.
You will find all information on:

\sphinxAtStartPar
\sphinxurl{https://github.com/Jusas/WatneyAstrometry}

\sphinxAtStartPar
and

\sphinxAtStartPar
\sphinxurl{https://watney-astrometry.net}

\sphinxstepscope


\subsection{Example setup for rig}
\label{\detokenize{install/example:example-setup-for-rig}}\label{\detokenize{install/example::doc}}
\sphinxAtStartPar
To give a realistic example I will explain my own setup. I build a portable
silver box, which is the housing for the DC / DC converters, the power supply,
the ethernet switch, the mount computer and all the switches and connectors
needed for attaching the silver box to the mount.

\sphinxAtStartPar
All other components are located directly on the telescope, so the is only an
ethernet connection and two 12 V power supply wires to the telescope.

\noindent{\hspace*{\fill}\sphinxincludegraphics{{setup}.png}\hspace*{\fill}}

\sphinxstepscope


\section{Configuring}
\label{\detokenize{config/index:configuring}}\label{\detokenize{config/index::doc}}
\sphinxstepscope


\subsection{Mount}
\label{\detokenize{config/mount/index:mount}}\label{\detokenize{config/mount/index::doc}}
\sphinxstepscope


\subsubsection{Settling Time / Waiting Time}
\label{\detokenize{config/mount/settlingTime:settling-time-waiting-time}}\label{\detokenize{config/mount/settlingTime::doc}}
\sphinxAtStartPar
To accommodate several different use cases MW4 implements additional waiting
times to the core settling time, which is implemented and user directly from the
mount computer. The following image shows the setting of this parameter, which
could be also set and altered through 10micron tools.

\noindent{\hspace*{\fill}\sphinxincludegraphics{{mountSettlingTime}.png}\hspace*{\fill}}

\sphinxAtStartPar
This settling time is valid for all slews and movements of your mount once set.
Please have a look to the 10micron spec where this behaviour has to be taken into
account. Nevertheless for the modeling part MW4 add two more parameters as the
modeling process need heavy movement of the mount. Therefore MW4 call these
parameters not settling time but waiting time. These parameters could be set
under the mount parameters:

\noindent{\hspace*{\fill}\sphinxincludegraphics{{waitingTime}.png}\hspace*{\fill}}

\sphinxAtStartPar
The working principle is as follows: MW4 initiates a slew. This command is run by
the mount computer and takes the internal settling time into account. This means
after the mount came to physical stop, the mount computer will send the signal
slew finished after this time period (upper image). This is the case in all used
cases and will applied also during modeling process.

\sphinxAtStartPar
For the modeling process MW4 \sphinxstylestrong{adds} a waiting time before moving on after slew,
which means waiting the addition set time before starting a next exposure (you
know that MW4 runs asynchron for slew, expose and plate solve to improve speed).
The wait is only applied during the modeling process.

\sphinxAtStartPar
Furthermore MW4 will differentiate if the mount starts and stops on the same
pierside or if the was a meridian flip of the mount. For both cases you could set
the waiting time.

\sphinxstepscope


\subsection{Dome}
\label{\detokenize{config/dome/index:dome}}\label{\detokenize{config/dome/index::doc}}
\sphinxstepscope


\section{Using functions}
\label{\detokenize{using/index:using-functions}}\label{\detokenize{using/index::doc}}
\sphinxstepscope


\subsection{Use Almanac}
\label{\detokenize{using/almanac/index:use-almanac}}\label{\detokenize{using/almanac/index::doc}}
\sphinxstepscope


\subsection{Use Environment}
\label{\detokenize{using/environment/index:use-environment}}\label{\detokenize{using/environment/index::doc}}
\sphinxstepscope


\subsection{Generate Build Points}
\label{\detokenize{using/buildpoints/index:generate-build-points}}\label{\detokenize{using/buildpoints/index::doc}}
\sphinxstepscope


\subsection{Update IERS data}
\label{\detokenize{using/iers/index:update-iers-data}}\label{\detokenize{using/iers/index::doc}}
\sphinxstepscope


\subsection{Use Minor Planet data}
\label{\detokenize{using/minorplanets/index:use-minor-planet-data}}\label{\detokenize{using/minorplanets/index::doc}}
\sphinxstepscope


\subsection{Run Model Task}
\label{\detokenize{using/modeling/index:run-model-task}}\label{\detokenize{using/modeling/index::doc}}
\sphinxstepscope


\subsection{Use Satellite}
\label{\detokenize{using/satellite/index:use-satellite}}\label{\detokenize{using/satellite/index::doc}}
\sphinxstepscope


\section{Architectural topics and math}
\label{\detokenize{architecture/index:architectural-topics-and-math}}\label{\detokenize{architecture/index::doc}}
\sphinxAtStartPar
Within these pages I would explain how and why I made the architecture decisions
for linking it with the 10micron mount computer. This might help for setting up or
just explain the behavior you experience when using MW4. I do this also as my
development documentation. There might be some faults and error in it. If you find
one, please let me know. I would like to get MW4 from it’s technical base as clean
as possible.

\sphinxstepscope


\subsection{Handling time}
\label{\detokenize{architecture/a_time:handling-time}}\label{\detokenize{architecture/a_time::doc}}
\sphinxAtStartPar
One basic definition is that MW4 will use at any time the clock of the mount
computer. Therefore MW4 polls julian date, difference utc \sphinxhyphen{} ut1, time sidereal.
This allows full sync for any calculation to be made. No time from computer to
mount is necessary, but could be done at any time (except during model build run).
The mount mostly use the julian date representation except for model build where a
local sidereal time (LST) is used. In this case MW4 just stores the value and feed
it back when the model is programmed. That’s the reason why you should not change
time during model run.

\noindent{\hspace*{\fill}\sphinxincludegraphics{{time}.png}\hspace*{\fill}}

\sphinxAtStartPar
One important difference between MW4 and Mount exists. As I use skyfield as on of
the frameworks with it’s units for Angle, Coords, Time etc. I have to take the
time definition of skyfield into account. Skyfield chooses TT (Terrestrial Time) as
it’s basic concept, whereas the mount uses UTC (Coordinated Universal Time) as
reference. TT is a modern astronomical time standard defined by the International
Astronomical Union. TT is distinct from the time scale often used as a basis for
civil purposes, UTC. TT is indirectly the basis of UTC, via International Atomic
Time (TAI).

\sphinxstepscope


\subsection{Precision of internal calculations}
\label{\detokenize{architecture/calculations:precision-of-internal-calculations}}\label{\detokenize{architecture/calculations::doc}}
\sphinxAtStartPar
MW4 is using for all calculations the skyfield (\sphinxurl{https://rhodesmill.org/skyfield/})
from Brandon Rhodes. As for the new command set offered with 10microns FW3.x it
needs to calculate the alt/az coordinates for a satellite track each second for
the entire track. As you would like to follow the as precise as possible I made
some comparisons between the internal calculations done in 10micron mount and the
results provided by skyfield.

\sphinxAtStartPar
In skyfield there is a chapter about satellite calculations and precision:
\sphinxurl{https://rhodesmill.org/skyfield/earth-satellites.html\#avoid-calling-the-observe-method}
Despite the fact that the observe method is expensive the difference in calulation
time for a 900 step track is on my computer 120ms (using more precise observe
method) to 7ms (using the less precise difference).

\sphinxAtStartPar
Brandon writes about it:
\begin{quote}

\sphinxAtStartPar
While satellite positions are only accurate to about a kilometer anyway,
accounting for light travel time only affected the position in this case by
less than an additional tenth of a kilometer. This difference is not
meaningful when compared to the uncertainty that is inherent in satellite
positions to begin with, so you should neglect it and simply subtract
GCRS\sphinxhyphen{}centered vectors instead as detailed above.
\end{quote}

\sphinxAtStartPar
Here the charts for NOAA 15 {[}B{]} at julian date JD=2459333.26498 for the transit
happening. The used TLE data was:

\begin{sphinxVerbatim}[commandchars=\\\{\}]
\PYG{n}{NOAA} \PYG{l+m+mi}{15} \PYG{p}{[}\PYG{n}{B}\PYG{p}{]}
\PYG{l+m+mi}{1} \PYG{l+m+mi}{25338}\PYG{n}{U} \PYG{l+m+mi}{98030}\PYG{n}{A}   \PYG{l+m+mf}{21104.44658620}  \PYG{l+m+mf}{.00000027}  \PYG{l+m+mi}{00000}\PYG{o}{\PYGZhy{}}\PYG{l+m+mi}{0}  \PYG{l+m+mi}{29723}\PYG{o}{\PYGZhy{}}\PYG{l+m+mi}{4} \PYG{l+m+mi}{0}  \PYG{l+m+mi}{9990}
\PYG{l+m+mi}{2} \PYG{l+m+mi}{25338}  \PYG{l+m+mf}{98.6888} \PYG{l+m+mf}{133.5239} \PYG{l+m+mi}{0011555} \PYG{l+m+mf}{106.3612} \PYG{l+m+mf}{253.8839} \PYG{l+m+mf}{14.26021970192127}
\end{sphinxVerbatim}

\sphinxAtStartPar
You could see the alt/az of the sat track.

\noindent{\hspace*{\fill}\sphinxincludegraphics{{sat_track}.png}\hspace*{\fill}}

\sphinxAtStartPar
the difference for altitude between 10micron and skyfield

\noindent{\hspace*{\fill}\sphinxincludegraphics{{sat_altitude}.png}\hspace*{\fill}}

\sphinxAtStartPar
the difference for azimuth between 10micron and skyfield

\noindent{\hspace*{\fill}\sphinxincludegraphics{{sat_azimuth}.png}\hspace*{\fill}}

\sphinxAtStartPar
the difference for right ascension between 10micron and skyfield

\noindent{\hspace*{\fill}\sphinxincludegraphics{{sat_ra}.png}\hspace*{\fill}}

\sphinxAtStartPar
the difference for declination between 10micron and skyfield

\noindent{\hspace*{\fill}\sphinxincludegraphics{{sat_dec}.png}\hspace*{\fill}}

\sphinxAtStartPar
There is a set of plots for another satellite, which shows the same behavior. The
used TLE data was:

\begin{sphinxVerbatim}[commandchars=\\\{\}]
\PYG{n}{RAAVANA}\PYG{o}{\PYGZhy{}}\PYG{l+m+mi}{1}
\PYG{l+m+mi}{1} \PYG{l+m+mi}{44329}\PYG{n}{U} \PYG{l+m+mi}{98067}\PYG{n}{QE}  \PYG{l+m+mf}{21134.29933328}  \PYG{l+m+mf}{.00044698}  \PYG{l+m+mi}{00000}\PYG{o}{\PYGZhy{}}\PYG{l+m+mi}{0}  \PYG{l+m+mi}{30736}\PYG{o}{\PYGZhy{}}\PYG{l+m+mi}{3} \PYG{l+m+mi}{0}  \PYG{l+m+mi}{9995}
\PYG{l+m+mi}{2} \PYG{l+m+mi}{44329}  \PYG{l+m+mf}{51.6342} \PYG{l+m+mf}{100.9674} \PYG{l+m+mi}{0004554} \PYG{l+m+mf}{122.3279} \PYG{l+m+mf}{237.8162} \PYG{l+m+mf}{15.74179130108776}
\end{sphinxVerbatim}

\sphinxAtStartPar
You could see the alt/az of the sat track.

\noindent{\hspace*{\fill}\sphinxincludegraphics{{sat2_track}.png}\hspace*{\fill}}

\sphinxAtStartPar
the difference for altitude between 10micron and skyfield

\noindent{\hspace*{\fill}\sphinxincludegraphics{{sat2_altitude}.png}\hspace*{\fill}}

\sphinxAtStartPar
the difference for azimuth between 10micron and skyfield

\noindent{\hspace*{\fill}\sphinxincludegraphics{{sat2_azimuth}.png}\hspace*{\fill}}

\sphinxAtStartPar
the difference for right ascension between 10micron and skyfield

\noindent{\hspace*{\fill}\sphinxincludegraphics{{sat2_ra}.png}\hspace*{\fill}}

\sphinxAtStartPar
the difference for declination between 10micron and skyfield

\noindent{\hspace*{\fill}\sphinxincludegraphics{{sat2_dec}.png}\hspace*{\fill}}

\sphinxAtStartPar
For all calculations is valid:
\begin{itemize}
\item {} 
\sphinxAtStartPar
they are using refraction correction with the same values.

\item {} 
\sphinxAtStartPar
the coordinates from 10micron are gathered with :TLEGEQJD\#, :TLEGAZJD\# commands

\item {} 
\sphinxAtStartPar
julian date is in UTC time system

\item {} 
\sphinxAtStartPar
10micron firmware 3.0.4

\item {} 
\sphinxAtStartPar
skyfield version 1.39

\end{itemize}

\sphinxstepscope


\section{Changelogs}
\label{\detokenize{changelog/index:changelogs}}\label{\detokenize{changelog/index::doc}}
\sphinxAtStartPar
The changelogs contains the user related function or environment updates. For a
detailed changes list, please refer to the commit list on GitHub.

\sphinxstepscope


\subsection{Ideas for the future}
\label{\detokenize{changelog/features:ideas-for-the-future}}\label{\detokenize{changelog/features::doc}}\begin{itemize}
\item {} 
\sphinxAtStartPar
implementing Baader dome control e.g. 10micron dome interface

\item {} 
\sphinxAtStartPar
receive video indi for sat tracking and horizon setup

\item {} 
\sphinxAtStartPar
data exchange to EKOS SQLite for Location and horizon through SQLite interface

\item {} 
\sphinxAtStartPar
satellite scheduler

\item {} 
\sphinxAtStartPar
adding simple model feature for rainbow 135

\item {} 
\sphinxAtStartPar
simple MPC data like satellites

\end{itemize}

\sphinxstepscope


\subsection{Beta versions of MW4}
\label{\detokenize{changelog/changelog:beta-versions-of-mw4}}\label{\detokenize{changelog/changelog::doc}}\begin{itemize}
\item {} 
\sphinxAtStartPar
none so far

\end{itemize}


\subsection{Released versions of MW4}
\label{\detokenize{changelog/changelog:released-versions-of-mw4}}

\subsubsection{Version 3.0}
\label{\detokenize{changelog/changelog:version-3-0}}
\sphinxAtStartPar
3.2.6
\begin{itemize}
\item {} 
\sphinxAtStartPar
add: support for INDI Pegasus Uranus Meteo sensor

\item {} 
\sphinxAtStartPar
add: wait time after slew finished before exposing

\item {} 
\sphinxAtStartPar
change: writing pointing coordinates to fits header from MW4 now

\item {} 
\sphinxAtStartPar
improve: add waiting time for image file save for NINA and SGPro

\item {} 
\sphinxAtStartPar
improve: logging for NINA / SGPro controlled cameras

\item {} 
\sphinxAtStartPar
improve: gain handling when missing values in camera settings

\item {} 
\sphinxAtStartPar
improve: lower the dome radius to 0.8m

\item {} 
\sphinxAtStartPar
fix: typos and some minor bugs

\end{itemize}

\sphinxAtStartPar
3.2.5
\begin{itemize}
\item {} 
\sphinxAtStartPar
improve: add more information to the log file seeing

\item {} 
\sphinxAtStartPar
improve: openweathermap data handling (API)

\item {} 
\sphinxAtStartPar
improve: add support for pegasus uranus meteo sensor

\end{itemize}

\sphinxAtStartPar
3.2.4
\begin{itemize}
\item {} 
\sphinxAtStartPar
add: support for astap D80 database

\item {} 
\sphinxAtStartPar
improve: more robust implementation against touptek drivers

\item {} 
\sphinxAtStartPar
improve: add more information to the log file seeing

\item {} 
\sphinxAtStartPar
improve: openweathermap data handling (API)

\end{itemize}

\sphinxAtStartPar
3.2.3
\begin{itemize}
\item {} 
\sphinxAtStartPar
fix: correct editing points, when slew path is not selected

\item {} 
\sphinxAtStartPar
improve: sort horizon points when loading a file

\end{itemize}

\sphinxAtStartPar
3.2.2
\begin{itemize}
\item {} 
\sphinxAtStartPar
change: switch from forecast to weather api on openweathermap

\end{itemize}

\sphinxAtStartPar
3.2.1
\begin{itemize}
\item {} 
\sphinxAtStartPar
fix: change humidity and dewpoint value in driver as there were mixed up

\end{itemize}

\sphinxAtStartPar
3.2.0
\begin{itemize}
\item {} 
\sphinxAtStartPar
add: editable mount settling time for 10micron box (UI change!)

\item {} 
\sphinxAtStartPar
add: waiting time used w/o  meridian flip

\item {} 
\sphinxAtStartPar
add: bring “keep scale” when doing exposeN

\item {} 
\sphinxAtStartPar
improve: some refactoring for speed

\item {} 
\sphinxAtStartPar
improve: watney checking allows for multiple sets in one directory

\end{itemize}

\sphinxAtStartPar
3.1.0

\sphinxAtStartPar
Version 3.1 brings aarch64 support for arm back if using the new installer 3.1
\begin{itemize}
\item {} 
\sphinxAtStartPar
add: support for aarch64 on raspi for python 3.8 \sphinxhyphen{} 3.10 (needs installer 3.1)

\item {} 
\sphinxAtStartPar
add: support for ASTAP new databases D50, D20, D05

\item {} 
\sphinxAtStartPar
improve: speedup launch if INDI server not ready

\item {} 
\sphinxAtStartPar
improve: support for catalina

\item {} 
\sphinxAtStartPar
improve: ParkPos with 2 digits precision

\item {} 
\sphinxAtStartPar
fix: download sources IERS

\item {} 
\sphinxAtStartPar
fix: switching UTC / local times

\item {} 
\sphinxAtStartPar
fix: seeing entries visibility upon startup

\end{itemize}

\sphinxAtStartPar
3.0.1
\begin{itemize}
\item {} 
\sphinxAtStartPar
fix: ASCOM cover: brightness status.

\item {} 
\sphinxAtStartPar
fix: ASCOM cover: setting / reading brightness / max brightness

\item {} 
\sphinxAtStartPar
fix: almanac: text for “rise” and “set” were mixed

\item {} 
\sphinxAtStartPar
fix: DNS resolving

\item {} 
\sphinxAtStartPar
improve: add a hint for optimal binning to keep reasonable image sizes

\item {} 
\sphinxAtStartPar
improve meteoblue behavior: correct text and undisplayed if disabled

\item {} 
\sphinxAtStartPar
improve minor planets selection: adding multiple selection by mouse

\item {} 
\sphinxAtStartPar
improve refraction: when selecting internal sensor, go to automatic

\end{itemize}

\sphinxAtStartPar
3.0.0

\sphinxAtStartPar
Version 3.0 is a major release! Please update with care!
No ARM7 support / ARM64 only Python 3.8 \sphinxhyphen{} 3.9
\begin{itemize}
\item {} 
\sphinxAtStartPar
add: GUI: all charts could be zoomed and panned

\item {} 
\sphinxAtStartPar
add: GUI: all tab menu entries could be customized in order and stored /reset

\item {} 
\sphinxAtStartPar
add: GUI: all open windows could be collected to visual area

\item {} 
\sphinxAtStartPar
add: GUI: separate window with big buttons are available

\item {} 
\sphinxAtStartPar
add: GUI: reduced GUI configurable for a simpler user interface

\item {} 
\sphinxAtStartPar
add: video: support for up to 4 external RTSP streams or local cameras

\item {} 
\sphinxAtStartPar
add: video: adding authentication to video streams

\item {} 
\sphinxAtStartPar
add: video: adding support for HTTP and HTTPS streams

\item {} 
\sphinxAtStartPar
add: almanac: now supports UTC / local time

\item {} 
\sphinxAtStartPar
add: almanac: support set/rise times moon

\item {} 
\sphinxAtStartPar
add: environment: integrate meteoblue.com seeing conditions

\item {} 
\sphinxAtStartPar
add: analyse: charts could show horizon and values for each point

\item {} 
\sphinxAtStartPar
add: analyse: alt / az charts with iso 2d contour error curves

\item {} 
\sphinxAtStartPar
add: audio: sound for connection lost and sat start tracking

\item {} 
\sphinxAtStartPar
add: model points: multiple variants for edit and move points

\item {} 
\sphinxAtStartPar
add: model points: set dither on celestial paths

\item {} 
\sphinxAtStartPar
add: model points: generate from actual used mount model

\item {} 
\sphinxAtStartPar
add: model points: existing model files could be loaded

\item {} 
\sphinxAtStartPar
add: model points: golden spiral with exact number of points

\item {} 
\sphinxAtStartPar
add: polar align: adding hint how to use the knobs measures right

\item {} 
\sphinxAtStartPar
add: plate solve: new watney astrometry solver for all platforms

\item {} 
\sphinxAtStartPar
add: hemisphere: selection of terrain file

\item {} 
\sphinxAtStartPar
add: hemisphere: show actual model error in background

\item {} 
\sphinxAtStartPar
add: hemisphere: edit horizon model much more efficient

\item {} 
\sphinxAtStartPar
add: hemisphere: show 2d contour error curve from actual model

\item {} 
\sphinxAtStartPar
add: hemisphere: move point with mouse around

\item {} 
\sphinxAtStartPar
add: dome: control azimuth move CW / CCW for INDI

\item {} 
\sphinxAtStartPar
add: satellites: all time values could be UTC or local time now

\item {} 
\sphinxAtStartPar
add: MPC / IERS: adding alternative server for download

\item {} 
\sphinxAtStartPar
add: measure: window has max 5 charts now (from 3)

\item {} 
\sphinxAtStartPar
add: measure: more values (time delta, focus, cooler power, etc.)

\item {} 
\sphinxAtStartPar
add: image: photometry functions (aberration, roundness, etc.)

\item {} 
\sphinxAtStartPar
add: image: tilt estimation like ASTAP does as rectangle and triangle

\item {} 
\sphinxAtStartPar
add: image: add flip H and flip V

\item {} 
\sphinxAtStartPar
add: image: show RA/DEC coordinates in image if image was solved

\item {} 
\sphinxAtStartPar
add: image: center mount pointing g to any point in image by mouse double click

\item {} 
\sphinxAtStartPar
add: image: center mount pointing to image center

\item {} 
\sphinxAtStartPar
add: image: support for reading XISF files (simple versions)

\item {} 
\sphinxAtStartPar
add: imaging: separate page for imaging stats now

\item {} 
\sphinxAtStartPar
add: imaging: stats: calcs for plate solvers (index files etc.)

\item {} 
\sphinxAtStartPar
add: imaging: stats: calcs for critical focus zones

\item {} 
\sphinxAtStartPar
add: drivers: polling timing for drivers could be set

\item {} 
\sphinxAtStartPar
add: drivers: game controller interface for mount and dome

\item {} 
\sphinxAtStartPar
add: system: support for python 3.10

\item {} 
\sphinxAtStartPar
add: help: local install of documentation in PDF format

\item {} 
\sphinxAtStartPar
add: profiles: automatic translation from v2.2.x to 3.x

\item {} 
\sphinxAtStartPar
improve: GUI: layout for main window optimized and consistent and wording updates

\item {} 
\sphinxAtStartPar
improve: GUI: complete rework of charting: performance and functions

\item {} 
\sphinxAtStartPar
improve: GUI: clean up and optimize IERS download messages

\item {} 
\sphinxAtStartPar
improve: GUI: get more interaction bullet prove for invalid cross use cases

\item {} 
\sphinxAtStartPar
improve: GUI: moved on / off mount to their settings: avoid undesired shutoff

\item {} 
\sphinxAtStartPar
improve: GUI: show twilight and moon illumination in main window

\item {} 
\sphinxAtStartPar
improve: INDI: correcting setting parameters on startup

\item {} 
\sphinxAtStartPar
improve: model points: optimized DSO path generation (always fit, less params)

\item {} 
\sphinxAtStartPar
improve: model run: refactoring

\item {} 
\sphinxAtStartPar
improve: model run: better information about status and result

\item {} 
\sphinxAtStartPar
improve: hemisphere: improve solved point presentation (white, red)

\item {} 
\sphinxAtStartPar
improve: plate solve: compatibility checks

\item {} 
\sphinxAtStartPar
improve: system: all log files will be stored in a separate folder /log

\item {} 
\sphinxAtStartPar
improve: system: enable usage of python 3.10

\item {} 
\sphinxAtStartPar
improve: system: use latest PyQt5 version

\item {} 
\sphinxAtStartPar
improve: system: adjust window sizes to be able to make mosaic layout on desktop

\item {} 
\sphinxAtStartPar
improve: system: moved to actual jpl kernel de440.bsp for ephemeris calcs

\item {} 
\sphinxAtStartPar
remove: system: matplotlib package and replace with more performant pyqtgraph

\item {} 
\sphinxAtStartPar
remove: system: PIL package and replace with more powerful cv2

\item {} 
\sphinxAtStartPar
remove: system: move from deprecated distutils to packaging

\item {} 
\sphinxAtStartPar
remove: system: support for python 3.7 as some libraries stopped support

\item {} 
\sphinxAtStartPar
remove: imageW: stacking in imageW as it was never used

\item {} 
\sphinxAtStartPar
remove: testing support for OSx Mojave and OSx Catalina (still should work)

\item {} 
\sphinxAtStartPar
fix: drivers: device selection tab was not properly positioned in device popup

\end{itemize}


\subsubsection{Version 2.2}
\label{\detokenize{changelog/changelog:version-2-2}}
\sphinxAtStartPar
2.2.9
\begin{itemize}
\item {} 
\sphinxAtStartPar
fix: internal updater shows only alpha versions instead of betas

\end{itemize}

\sphinxAtStartPar
2.2.8
\begin{itemize}
\item {} 
\sphinxAtStartPar
fix: updates for supporting newer ASTAP versions

\item {} 
\sphinxAtStartPar
fix: model run will cancel if solving fails

\item {} 
\sphinxAtStartPar
fix: workaround ASTAP FITS outputs which are not readable via astropy

\item {} 
\sphinxAtStartPar
update ephemeris file

\end{itemize}

\sphinxAtStartPar
2.2.7
\begin{itemize}
\item {} 
\sphinxAtStartPar
fix: text labels

\item {} 
\sphinxAtStartPar
fix: getting min / max values from indi devices

\item {} 
\sphinxAtStartPar
fix: updates for supporting newer ASTAP versions

\item {} 
\sphinxAtStartPar
fix: model run will cancel if solving fails

\end{itemize}

\sphinxAtStartPar
2.2.6
\begin{itemize}
\item {} 
\sphinxAtStartPar
fix: reduce load in debug trace mode

\item {} 
\sphinxAtStartPar
fix: model process stalls in some cases in normal mode

\item {} 
\sphinxAtStartPar
fix: text labels

\item {} 
\sphinxAtStartPar
fix: getting min / max values from indi devices

\end{itemize}

\sphinxAtStartPar
2.2.5
\begin{itemize}
\item {} 
\sphinxAtStartPar
fix: reduce load in debug trace mode

\item {} 
\sphinxAtStartPar
fix: model process stalls in some cases in normal mode

\end{itemize}

\sphinxAtStartPar
2.2.4
\begin{itemize}
\item {} 
\sphinxAtStartPar
fix: remove race condition for large image file causing solve error in ASTAP

\item {} 
\sphinxAtStartPar
fix: reduce load in debug trace mode

\end{itemize}

\sphinxAtStartPar
2.2.3
\begin{itemize}
\item {} 
\sphinxAtStartPar
fix: mount orientation in southern hemisphere

\end{itemize}

\sphinxAtStartPar
2.2.2
\begin{itemize}
\item {} 
\sphinxAtStartPar
fix: almanac moon phase drawing error

\end{itemize}

\sphinxAtStartPar
2.2.1
\begin{itemize}
\item {} 
\sphinxAtStartPar
update: builtin data for finals200.all

\item {} 
\sphinxAtStartPar
fix: download iers data: fix file not found feedback

\end{itemize}

\sphinxAtStartPar
2.2.0
\begin{itemize}
\item {} 
\sphinxAtStartPar
add: support SGPro camera as device

\item {} 
\sphinxAtStartPar
add: support N.I.N.A. camera as device

\item {} 
\sphinxAtStartPar
add: two modes for SGPro and N.I.N.A.: App or MW4 controlled

\item {} 
\sphinxAtStartPar
add: debayer (4 modes) all platforms (armv7, StellarMate, Astroberry)

\item {} 
\sphinxAtStartPar
add: filter satellites for twilight visibility settings

\item {} 
\sphinxAtStartPar
add: setting performance for windows automation (slow / normal / fast)

\item {} 
\sphinxAtStartPar
add: auto abort imaging when camera device is disconnected

\item {} 
\sphinxAtStartPar
add: missing cursor in virtual keypad window

\item {} 
\sphinxAtStartPar
add: support for keyboard usage in virtual keypad window

\item {} 
\sphinxAtStartPar
add: screenshot as PNG save for actual window with key F5

\item {} 
\sphinxAtStartPar
add: screenshots as PNG save for all open windows with key F6

\item {} 
\sphinxAtStartPar
add: query DSO objects for DSO path setting in build model

\item {} 
\sphinxAtStartPar
improved: flexible satellite handling when mount not connected

\item {} 
\sphinxAtStartPar
improved: show selected satellite name in satellite windows title

\item {} 
\sphinxAtStartPar
improved: 3D simulator drawing

\item {} 
\sphinxAtStartPar
improved: updater now avoids installation into system package

\item {} 
\sphinxAtStartPar
improved: GUI for imaging tab \sphinxhyphen{} disable all invalid interfaces

\item {} 
\sphinxAtStartPar
improved: redesign analyse window to get more space for further charts

\item {} 
\sphinxAtStartPar
improved: Tools: move mount: better UI, tooltips, multi steps in alt/az

\item {} 
\sphinxAtStartPar
improved: gui in image window when displaying different types

\item {} 
\sphinxAtStartPar
improved: reduced memory consumption if display raw images

\item {} 
\sphinxAtStartPar
improved: defining park positions with digit and improve gui for buttons

\item {} 
\sphinxAtStartPar
improved: when pushbutton shows running, invert icons as well

\item {} 
\sphinxAtStartPar
improve: moon phases in different color schemes

\item {} 
\sphinxAtStartPar
upgrade: pywin32 library to version 303 (windows)

\item {} 
\sphinxAtStartPar
upgrade: skyfield library to 1.41

\item {} 
\sphinxAtStartPar
upgrade: numpy library to 1.21.4

\item {} 
\sphinxAtStartPar
upgrade: matplotlib to 3.5.1

\item {} 
\sphinxAtStartPar
upgrade: scipy library to 1.7.3

\item {} 
\sphinxAtStartPar
upgrade requests library to 2.27.2

\item {} 
\sphinxAtStartPar
upgrade importlib\_metadata library to 4.10.0

\item {} 
\sphinxAtStartPar
upgrade deepdiff library to 5.7.0

\item {} 
\sphinxAtStartPar
upgrade wakeonlan library to 2.1.0

\item {} 
\sphinxAtStartPar
upgrade pybase64 library to 1.2.1

\item {} 
\sphinxAtStartPar
upgrade websocket\sphinxhyphen{}client library to 1.2.3

\item {} 
\sphinxAtStartPar
fix: simulator in southern hemisphere

\end{itemize}


\subsubsection{Version 2.1}
\label{\detokenize{changelog/changelog:version-2-1}}
\sphinxAtStartPar
2.1.7
\begin{itemize}
\item {} 
\sphinxAtStartPar
add: 12 build point option for model generation

\item {} 
\sphinxAtStartPar
add: grouping updater windows upper left corner

\item {} 
\sphinxAtStartPar
add: support for languages other than english in automation

\item {} 
\sphinxAtStartPar
add: minimize cmd window once MW4 is started

\item {} 
\sphinxAtStartPar
fix: KMTronic Relay messages

\end{itemize}

\sphinxAtStartPar
2.1.6
\sphinxhyphen{} add: explicit logging of automation windows strings for debug
\sphinxhyphen{} add: showing now detected updater path and app
\sphinxhyphen{} revert: fixes for german as they do not work

\sphinxAtStartPar
2.1.5
\begin{itemize}
\item {} 
\sphinxAtStartPar
fix: checking windows python version for automation

\end{itemize}

\sphinxAtStartPar
2.1.4
\begin{itemize}
\item {} 
\sphinxAtStartPar
add: enabled internal updater for astroberry and stellarmate

\item {} 
\sphinxAtStartPar
add: temperature measurement for camera

\item {} 
\sphinxAtStartPar
improved: logging for ASCOM threading

\item {} 
\sphinxAtStartPar
improved: image handling

\item {} 
\sphinxAtStartPar
fix: DSLR camera devices

\end{itemize}

\sphinxAtStartPar
2.1.3
\begin{itemize}
\item {} 
\sphinxAtStartPar
add: config adjustments for astroberry and stellarmate devices (no debayer)

\item {} 
\sphinxAtStartPar
improved: logging for UI events

\end{itemize}

\sphinxAtStartPar
2.1.2
\begin{itemize}
\item {} 
\sphinxAtStartPar
fix: non connected mount influences camera on ASCOM / ALPACA

\item {} 
\sphinxAtStartPar
fix: logging string formatting

\end{itemize}

\sphinxAtStartPar
2.1.1
\begin{itemize}
\item {} 
\sphinxAtStartPar
fix: for arm64 only: corrected import for virtual keypad

\item {} 
\sphinxAtStartPar
fix: arrow keys on keypad did accept long mouse press

\end{itemize}

\sphinxAtStartPar
2.1.0
\begin{itemize}
\item {} 
\sphinxAtStartPar
add: hemisphere window: help for choosing the right star for polar alignment

\item {} 
\sphinxAtStartPar
add: hemisphere terrain adjust for altitude of image beside azimuth

\item {} 
\sphinxAtStartPar
add: angular error ra / dec axis in measurement

\item {} 
\sphinxAtStartPar
add: device connection similar for ASCOM and ALPACA devices

\item {} 
\sphinxAtStartPar
add: extended satellite search and filter capabilities (spreadsheet style)

\item {} 
\sphinxAtStartPar
add: estimation of satellite apparent magnitude

\item {} 
\sphinxAtStartPar
add: extended satellite tracking and tuning capabilities

\item {} 
\sphinxAtStartPar
add: enabling loading a custom satellite TLE data file

\item {} 
\sphinxAtStartPar
add: command window for manual mount commands

\item {} 
\sphinxAtStartPar
add: sorting for minimal dome slew in build point selection

\item {} 
\sphinxAtStartPar
add: setting prediction time of almanac (shorter reduces cpu load)

\item {} 
\sphinxAtStartPar
add: providing 3 different color schemes

\item {} 
\sphinxAtStartPar
add: virtual keypad available for RPi 3/4 users now

\item {} 
\sphinxAtStartPar
improve: check if satellite data is valid (avoid error messages)

\item {} 
\sphinxAtStartPar
improve: better hints when using 10micron updater

\item {} 
\sphinxAtStartPar
improve: simplified signals generation

\item {} 
\sphinxAtStartPar
improve: analyse window plots

\item {} 
\sphinxAtStartPar
improve: rewrite alpaca / ascom interface

\item {} 
\sphinxAtStartPar
improve: gui for running functions

\item {} 
\sphinxAtStartPar
improve: test coverage

\item {} 
\sphinxAtStartPar
remove: push time from mount to computer: in reliable and unstable

\item {} 
\sphinxAtStartPar
fix: segfault in qt5lib on ubuntu

\end{itemize}


\subsubsection{Version 2.0}
\label{\detokenize{changelog/changelog:version-2-0}}
\sphinxAtStartPar
2.0.6
\begin{itemize}
\item {} 
\sphinxAtStartPar
fixes

\end{itemize}

\sphinxAtStartPar
2.0.5
\begin{itemize}
\item {} 
\sphinxAtStartPar
fix: bug when running “stop exposure” in ASCOM

\end{itemize}

\sphinxAtStartPar
2.0.4
\begin{itemize}
\item {} 
\sphinxAtStartPar
improvement: GUI for earth rotation data update, now downloads

\item {} 
\sphinxAtStartPar
improvement: performance for threads.

\item {} 
\sphinxAtStartPar
improvement: added FITS header entries for ALPACA and ASCOM

\item {} 
\sphinxAtStartPar
fix: removed stopping DAT when starting model

\end{itemize}

\sphinxAtStartPar
2.0.3
\begin{itemize}
\item {} 
\sphinxAtStartPar
improvement: GUI for earth rotation data update, now downloads

\item {} 
\sphinxAtStartPar
improvement: performance for threads.

\end{itemize}

\sphinxAtStartPar
2.0.2
\begin{itemize}
\item {} 
\sphinxAtStartPar
fix: robustness against errors in ALPACA server due to memory faults \#174

\item {} 
\sphinxAtStartPar
fix: robustness against filter names / numbers from ALPACA server \#174

\item {} 
\sphinxAtStartPar
fix: cleanup import for pywinauto timings import \#175

\item {} 
\sphinxAtStartPar
improvement: avoid meridian flip \#177

\item {} 
\sphinxAtStartPar
improvement: retry numbers as int \#178

\end{itemize}

\sphinxAtStartPar
2.0.1
\begin{itemize}
\item {} 
\sphinxAtStartPar
fix: MW4 not shutting down when dome configured, but not connected

\item {} 
\sphinxAtStartPar
fix mirrored display of points in polar hemisphere view

\end{itemize}

\sphinxAtStartPar
2.0.0
\begin{itemize}
\item {} 
\sphinxAtStartPar
add new updater concept

\item {} 
\sphinxAtStartPar
add mount clock sync feature

\item {} 
\sphinxAtStartPar
add simulator feature

\item {} 
\sphinxAtStartPar
add terrain image feature

\item {} 
\sphinxAtStartPar
add dome following when mount is in satellite tracking mode

\item {} 
\sphinxAtStartPar
add dome dynamic following feature: reduction of slews for dome

\item {} 
\sphinxAtStartPar
add setting label support for UPB dew entries

\item {} 
\sphinxAtStartPar
add auto dew control support for Pegasus UPB

\item {} 
\sphinxAtStartPar
add switch support for ASCOM/ALPACA Pegasus UPB

\item {} 
\sphinxAtStartPar
add observation condition support for ASCOM/ALPACA Pegasus UPB

\item {} 
\sphinxAtStartPar
add feature for RA/DEC FITS writing for INDI server without snooping

\item {} 
\sphinxAtStartPar
add completely revised satellite tracking menu gui

\item {} 
\sphinxAtStartPar
add partially satellite tracking before / after possible flip

\item {} 
\sphinxAtStartPar
add satellite track respect horizon line and meridian limits

\item {} 
\sphinxAtStartPar
add tracking simulator feature to test without waiting for satellite

\item {} 
\sphinxAtStartPar
add alt/az pointer to satellite view

\item {} 
\sphinxAtStartPar
add reverse order for failed build point retry

\item {} 
\sphinxAtStartPar
add automatic enable webinterface for keypad use

\item {} 
\sphinxAtStartPar
add broadcast address and port for WOL

\item {} 
\sphinxAtStartPar
add new IERS and lead second download

\item {} 
\sphinxAtStartPar
add more functions are available without mount connected

\item {} 
\sphinxAtStartPar
add change mouse pointer in hemisphere

\item {} 
\sphinxAtStartPar
add offset and gain setting to imaging

\item {} 
\sphinxAtStartPar
add disable model point edit during model build run

\item {} 
\sphinxAtStartPar
update debug standard moved from WARN to INFO

\item {} 
\sphinxAtStartPar
update underlying libraries

\item {} 
\sphinxAtStartPar
update GUI improvements

\item {} 
\sphinxAtStartPar
fix for INDI cameras sending two times busy and exposure=0

\item {} 
\sphinxAtStartPar
fix slewing message dome when disconnected

\item {} 
\sphinxAtStartPar
fix retry mechanism for failed build points

\item {} 
\sphinxAtStartPar
fix using builtins for skyfield and rotation update

\item {} 
\sphinxAtStartPar
fix plate solve sync function

\end{itemize}


\subsubsection{Version 1.1}
\label{\detokenize{changelog/changelog:version-1-1}}
\sphinxAtStartPar
1.1.1
\begin{itemize}
\item {} 
\sphinxAtStartPar
adding fix for INDI cameras sending two times BUSY, EXP=0

\end{itemize}

\sphinxAtStartPar
1.1.0
\begin{itemize}
\item {} 
\sphinxAtStartPar
adding release notes showing new capabilities in message window

\item {} 
\sphinxAtStartPar
adding cover light on / off

\item {} 
\sphinxAtStartPar
adding cover light intensity settings

\item {} 
\sphinxAtStartPar
reversing E/W for polar diagram in hemisphere window

\item {} 
\sphinxAtStartPar
adding push mount time to computer manual / hourly

\item {} 
\sphinxAtStartPar
adding contour HFD plot to image windows

\item {} 
\sphinxAtStartPar
adding virtual emergency stop key on time group

\item {} 
\sphinxAtStartPar
update build\sphinxhyphen{}in files if newer ones are shipped

\item {} 
\sphinxAtStartPar
auto restart MW4 after update

\item {} 
\sphinxAtStartPar
adding OBJCTRA / OBJCTDEC keywords when reading FITs

\item {} 
\sphinxAtStartPar
upgrade various libraries

\end{itemize}

\sphinxstepscope


\subsection{Released versions of scripts}
\label{\detokenize{changelog/changelogScripts:released-versions-of-scripts}}\label{\detokenize{changelog/changelogScripts::doc}}
\sphinxAtStartPar
The changelog contains the user related function or environment updates. For a
detailed changes list, please refer to the commit list on GitHub.


\subsubsection{Unreleased versions of scripts}
\label{\detokenize{changelog/changelogScripts:unreleased-versions-of-scripts}}

\subsubsection{Released versions of scripts}
\label{\detokenize{changelog/changelogScripts:id1}}
\sphinxAtStartPar
3.3
\begin{itemize}
\item {} 
\sphinxAtStartPar
adding verbose option to get outputs during install

\end{itemize}

\sphinxAtStartPar
3.2
\begin{itemize}
\item {} 
\sphinxAtStartPar
remove support for win32

\end{itemize}

\sphinxAtStartPar
3.1
\begin{itemize}
\item {} 
\sphinxAtStartPar
enable back aarch64 support for arm

\end{itemize}

\sphinxAtStartPar
3.0
\begin{itemize}
\item {} 
\sphinxAtStartPar
enable all feature of scripts to one single startup.pyz file with parameters

\item {} 
\sphinxAtStartPar
new script is limited to windows, mac and linux only. No support of Raspi or
aarch64

\end{itemize}

\sphinxAtStartPar
2.4
\begin{itemize}
\item {} 
\sphinxAtStartPar
windows only: improved robustness against local installs

\item {} 
\sphinxAtStartPar
adding a clean system utility

\end{itemize}

\sphinxAtStartPar
2.3
\begin{itemize}
\item {} 
\sphinxAtStartPar
windows only: improved robustness against local installs

\end{itemize}

\sphinxAtStartPar
2.2
\begin{itemize}
\item {} 
\sphinxAtStartPar
adding support for astroberry and stellar mate (based on astroberry?)

\item {} 
\sphinxAtStartPar
adding workaround on windows for removed python2 support in setuptools

\end{itemize}

\sphinxAtStartPar
2.1
\begin{itemize}
\item {} 
\sphinxAtStartPar
fixes

\end{itemize}

\sphinxAtStartPar
2.0
\begin{itemize}
\item {} 
\sphinxAtStartPar
update to requirements v2.0beta release of MW4

\end{itemize}

\sphinxAtStartPar
1.1
\begin{itemize}
\item {} 
\sphinxAtStartPar
update to requirements v1.1 release of MW4

\end{itemize}

\sphinxAtStartPar
0.4
\begin{itemize}
\item {} 
\sphinxAtStartPar
support for python 3.7 \sphinxhyphen{} 3.9, removing support 3.6

\item {} 
\sphinxAtStartPar
added prebuild wheels for ubuntu mate 18.04 aarch64 for python 3.7 \sphinxhyphen{} 3.9

\item {} 
\sphinxAtStartPar
added prebuild wheels for ubuntu mate 20.04 aarch64 for python 3.7 \sphinxhyphen{} 3.9

\end{itemize}

\sphinxAtStartPar
0.3
\begin{itemize}
\item {} 
\sphinxAtStartPar
adding more error log support

\item {} 
\sphinxAtStartPar
improved ui

\end{itemize}

\sphinxAtStartPar
0.2
\begin{itemize}
\item {} 
\sphinxAtStartPar
added support functions for better error handling while using

\end{itemize}

\sphinxAtStartPar
0.1
\begin{itemize}
\item {} 
\sphinxAtStartPar
adding support for raspberryPi3 with ubuntu mate

\item {} 
\sphinxAtStartPar
adding support for raspberryPi4 with ubuntu mate server / kubuntu

\end{itemize}

\sphinxAtStartPar
0.0
\begin{itemize}
\item {} 
\sphinxAtStartPar
adding support for automatic installation on windows

\item {} 
\sphinxAtStartPar
adding support for automatic installation on ubuntu

\item {} 
\sphinxAtStartPar
adding support for automatic installation on osx

\end{itemize}



\renewcommand{\indexname}{Index}
\printindex
\end{document}